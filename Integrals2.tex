\documentclass[a4paper]{article}
\usepackage{amsmath}
\usepackage{amssymb}
\usepackage{hyphenat}
\usepackage{graphicx}
\usepackage{pgfplots}
\usepackage[skip=10pt]{parskip}
\usepackage{enumitem}
\usepackage [spanish]{babel}
\usepackage [autostyle, english = american]{csquotes}
\MakeOuterQuote{"}

\pgfplotsset{compat = newest}
\begin{document}
\pretolerance=2000
\section{Funções de mais de uma variável}
\subsection*{Introdução}
\par Vimos que uma função $f:\mathbb{R}\rightarrow\mathbb{R}$ é uma \textit{regra} que recebe uma entrada, $x\in\mathbb{R}$, e nos dá uma única saída, $f(x)\in \mathbb{R}.$
\par Agora voltamos nossa atenção para funções de duas variáveis, ou seja, funções onde a entrada consiste em um par de números, $(x,y)\in{R}^2$, e cuja saída é um número único $f(x,y)\in{R}.$
\par Em particular, estaremos principalmente preocupados com funções de duas variáveis onde as variáveis são \textit{independentes}, ou seja, o valor de \textbf{x} pode ser escolhido independentemente do valor de \textbf{y} e vice-versa. Como veremos, as funções de duas variáveis ocorrem frequentemente na economia e em outros campos em que desejamos aplicar técnicas matemáticas. Dois exemplos importantes de tais funções da economia são:
\begin{itemize}
    \item A \textit{função de produção} de uma empresa, \textit{$\textbf{q(k,l)}$}, fornece a quantidade que ela produz ao usar k unidades de capital e l unidades de trabalho.
    \item A \textit{função de utilidade} de um consumidor, \textit{$\textbf{u(x\textsubscript{1},x\textsubscript{2})}$}, descreve quanta \textit{utilidade} um consumidor deriva de um bundle \textit{\textbf{(x1,x2)}} de dois bens. Como tal, permite-nos comparar as preferências do consumidor quando confrontado com diferentes combinações destes dois bens.
\end{itemize}
\par Observe que a teoria que consideramos aqui se estende ao caso geral em que a entrada consiste em \textbf{n} números \textit{\textbf{(x\textsubscript{1},x\textsubscript{2},…,x\textsubscript{n})}}. Esta extensão para funções de \textbf{n} variáveis $(n \geq 3)$  deve ser óbvia e por isso não gastamos muito tempo com isso aqui. No entanto, embora estejamos lidando principalmente com o caso de duas variáveis, ocasionalmente consideraremos funções de mais de duas variáveis.
\subsection*{Funções de duas variáveis}
\par Seja $f:\mathbb{R\textsuperscript{2}}\to\mathbb{R}$ uma função das duas variáveis independentes \textbf{x} e \textbf{y}. Podemos pensar em qualquer entrada \textit{\textbf{(a,b)}} como um ponto no plano \textit{\textbf{(x,y)}} e a saída será o valor correspondente de \textit{\textbf{(f)}}, ou seja, \textit{\textbf{f(a,b)}}, que podemos tomar como o número \textbf{c}.
\par Ou seja, em geral, cada ponto \textit{\textbf{(x,y)}} no plano \textit{\textbf{(x,y)}} terá uma saída dada pelo valor correspondente de $f$, ou seja, \textit{\textbf{f(x,y)}}, que podemos tomar como o valor de outra variável z. Como tal, para visualizar uma função de duas variáveis, precisamos de três eixos, dois para representar as entradas, ou seja, \textit{\textbf{x}} e \textit{\textbf{y}}, e um para representar a saída, ou seja, \textit{\textbf{z}}.
\par Se fizermos isso para todas as entradas possíveis \textit{\textbf{$(x,y)\in\mathbb{R}\textsuperscript{2}$}} obtemos uma superfície no espaço tridimensional cuja equação é dada por \textit{\textbf{$z = f(x,y)$}}.
\subsection*{Planos}
\par O tipo mais simples de função de duas variáveis é aquele que é linear em \textit{\textbf{x}} e \textit{\textbf{y}}, ou seja, onde
    \begin{equation}
        f(x,y) = ax + by
    \end{equation}
para algumas constantes \textit{\textbf{a}} e \textit{\textbf{b}}. Tais funções representam planos e, em geral, qualquer superfície que tenha uma equação da forma
    \begin{equation}
        ax + by + cz = d
    \end{equation}
onde pelo menos uma das constantes \textit{\textbf{a}}, \textit{\textbf{b}} e \textit{\textbf{c}} é diferente de zero representará um plano. Os tipos importantes de planos são, basicamente, aqueles que se enquadram nas seguintes categorias:
\begin{itemize}
    \item Os planos \textit{\textbf{(x,y)}}, \textit{\textbf{(y,z)}} e \textit{\textbf{(z,x)}} que têm equações z=0, x=0 e y=0 respectivamente.(Estes são os planos ilustrados abaixo em (a), (b) e (c), respectivamente.)
    \item Planos paralelos aos planos \textit{\textbf{(x,y)}}, \textit{\textbf{(y,z)}} e \textit{\textbf{(z,x)}} que, para alguma constante \textit{\textbf{c}}, terão as equações $z = c$, $x = c$ e $y = c$ respectivamente. (Estes são os outros planos ilustrados abaixo em (a), (b) e (c), respectivamente.)
\end{itemize}

\begin{figure}[ht]
    \includegraphics[width=\linewidth]{fig1.png}
    \caption{(a), (b), (c)}
    \label{fig1}
\end{figure}
\section{Planos horizontais e contornos}
\subsection*{Introdução}
\par Uma maneira de visualizar uma superfície é observar seus contornos, que são as curvas de interseção que surgem quando observamos os pontos de interseção de uma superfície com planos paralelos ao plano \textit{\textbf{(x,y)}}.

Para encontrar os contornos, tomamos um plano paralelo ao plano \textit{\textbf{(x,y)}}, digamos o plano $z = c$, e encontramos a curva de interseção entre ele e a superfície $z = f(x,y)$, ou seja, a curva com a equação $c = f(x,y)$. Esta curva é o contorno $z = c$, ou seja, o conjunto de pontos \textit{\textbf{(x,y)}} que dão $z = c$ quando os colocamos na equação $z = f(x,y)$.
\subsubsection*{Exemplo}
O contorno $z = 2$ da superfície $z = x - y + 4$. E também para $z = 4$ e $z = 6$.
\par Para encontrar o contorno $z = 2$ da superfície $z = x - y + 4$ precisamos encontrar a curva de interseção, que neste caso é dada por
    \begin{equation}
         2 = x - y + 4
    \end{equation}
\par Reorganizando isso dá a equação $y = x + 2$ que é a equação de uma linha reta.
Da mesma forma, encontramos que:
\begin{itemize}
    \item Para $z = 4$, a curva de interseção é dada por $4 = x - y + 4$ que nos dá $y = x$.

    \item Para $z = 6$, a curva de interseção é dada por $6 = x - y + 4$ que nos dá $y = x - 2$.
\end{itemize}
Assim, vemos a partir dessas equações que esses dois contornos são linhas retas também.

\begin{figure}[ht]
    \centering
\begin{tikzpicture}
    \begin{axis}[
        grid = both,
        major grid style = {lightgray},
        minor grid style = {lightgray!25},
    ]
            \addplot[
                domain = 0:30,
                samples = 200,
                smooth,
                thick,
                red,
            ] {x + 2};
            \addplot[
                domain = 0:30,
                samples = 200,
                smooth,
                thick,
                green,
            ] {x};
            \addplot[
                domain = 0:30,
                samples = 200,
                smooth,
                thick,
                blue,
            ] {x - 2};
            \legend{
                x + 2, 
                x,
                x - 2,
            }
    \end{axis}
\end{tikzpicture}
\end{figure}
\subsubsection*{Exemplo 2}
Encontre as equações dos contornos $z = -10$, $z = 0$ e $z = 10$ da superfície $z = 4x + 2y - 2$ e esboce-as no plano \textit{\textbf{(x,y)}} rotulando claramente o valor de z que está associado a cada contorno.
\begin{figure}[h!]
    \centering
\begin{tikzpicture}
    \begin{axis}[
        grid = both,
        major grid style = {lightgray},
        minor grid style = {lightgray!25},
    ]
            \addplot[
                domain = 0:30,
                samples = 200,
                smooth,
                thick,
                red,
            ] {-2*x - 4};
            \addplot[
                domain = 0:30,
                samples = 200,
                smooth,
                thick,
                green,
            ] {-2*x + 1};
            \addplot[
                domain = 0:30,
                samples = 200,
                smooth,
                thick,
                blue,
            ] {-2*x + 6};
            \legend{
                -2x - 4, 
                -2x + 1,
                -2x + 6,
            }
    \end{axis}
\end{tikzpicture}
\end{figure}
\subsubsection*{Exemplo 3}
Para encontrar o contorno $z = 16$ da superfície $z = x\textsuperscript2 + y\textsuperscript2$ precisamos encontrar a curva de interseção que, neste caso, é simplesmente
    \begin{equation}
         16 = x\textsuperscript2 + y\textsuperscript2
    \end{equation}
Esta é a equação de um círculo, centrado na origem, com um raio de quatro.
Para encontrar os contornos $z = c$ nos três casos indicados, basta descobrir qual é a curva nos três casos.
\par Então, temos o seguinte.
\begin{itemize}
    \item Se $c > 0$, o contorno é um círculo, centrado na origem, com raio $\sqrt{c}$.
    \item Se $c = 0$, o contorno é o ponto \textbf{(0,0)} pois esta é a única solução para a equação $x\textsuperscript2 + y\textsuperscript2 = 0$.
    \item Se $c < 0$, não há contornos, pois sabemos que $x\textsuperscript2 + y\textsuperscript2 \geq 0$ para todos os valores de \textbf{x} e \textbf{y}.
\end{itemize}
Em particular, observe que $z = 0$ é o menor valor de \textbf{z} que surge de um ponto nesta superfície.
\section{Planos verticais e as seções de uma superfície}
\subsection*{Introdução}
Outra maneira de visualizar uma superfície é observar suas seções, que são as curvas de interseção que surgem quando observamos os pontos de interseção de uma superfície com planos perpendiculares ao plano \textit{\textbf{(x,y)}}. Para encontrar as seções, tomamos um plano perpendicular ao plano \textit{\textbf{(x,y)}} e encontramos a curva de interseção entre ele e a superfície $z = f(x,y)$. Em particular, neste curso, só precisaremos considerar seções que surgem de planos que são paralelos ao plano \textit{\textbf{(x,z)}} (ou seja, $y = c$ para alguma constante \textbf{(c)}) ou paralelos ao plano \textit{\textbf{(y,z)}} (isto é, $x = c$ para alguma constante \textbf{(c)}).
\par Como tal, as seções mais fáceis de esboçar são aquelas que obtemos quando consideramos os planos \textit{\textbf{(x,z)}} e \textit{\textbf{(y,z)}} que são ambos perpendiculares ao plano \textit{\textbf{(x,y)}}. Em particular, descobrimos que a seção que obtemos do:
    \begin{itemize}
        \item Plano \textit{\textbf{(x,z)}}, que tem a equação $y = 0$, é a curva de interseção entre ele e a superfície $z = f(x,y)$, ou seja, a curva com a equação $z = f(x,0)$.
        \item Plano \textit{\textbf{(y,z)}}, que tem a equação $x = 0$, é a curva de interseção entre ele e a superfície $z = f(x,y)$, ou seja, a curva com a equação $z = f(0,y)$.
    \end{itemize}
    Vejamos como são essas seções no caso das duas superfícies que consideramos acima quando estávamos procurando por contornos.
\subsubsection*{Exemplo}
Para encontrar as seções da superfície $z = x - y + 4$ precisamos encontrar as curvas de interseção, que neste caso são dadas pelo seguinte.
\begin{itemize}
    \item Para a seção \textit{\textbf{(x,z)}}, temos $y = 0$ e assim a curva de interseção é dada por $z = x + 4$ e esta é uma linha reta no plano \textit{\textbf{(x,z)}}.
    \item Para a seção \textit{\textbf{(y,z)}}, temos $x = 0$ e assim a curva de interseção é dada por $z = -y + 4$ e esta é uma linha reta no plano \textit{\textbf{(y,z)}}.
\end{itemize}
\begin{figure}[ht]
    \centering
\begin{tikzpicture}
    \begin{axis}[
        grid = both,
        major grid style = {lightgray},
        minor grid style = {lightgray!25},
    ]
            \addplot[
                domain = -10:10,
                samples = 200,
                smooth,
                thick,
                red,
            ] {x + 4};
            \addplot[
                domain = -10:10,
                samples = 200,
                smooth,
                thick,
                green,
            ] {-x + 4};
            \legend{
                x + 4, 
                -y + 4,   
            }
    \end{axis}
\end{tikzpicture}
\end{figure}
\subsubsection*{Exemplo 2}
Para encontrar as seções da superfície $z = 4x + 2y - 2$ precisamos encontrar as curvas de interseção, que neste caso são dadas pelo seguinte.
\begin{itemize}
    \item Para a seção \textit{\textbf{(x,z)}}, temos $y = 0$ e assim a curva de interseção é dada por $z = 4x - 2$ e esta é uma linha reta no plano \textit{\textbf{(x,z)}}.
    \item Para a seção \textit{\textbf{(y,z)}}, temos $x = 0$ e assim a curva de interseção é dada por $z = 2y - 2$ e esta é uma linha reta no plano \textit{\textbf{(y,z)}}.
\end{itemize}
\begin{figure}[h!]
    \centering
    \begin{tikzpicture}
    \begin{axis}[
        grid = both,
        major grid style = {lightgray},
        minor grid style = {lightgray!25},
    ]
            \addplot[
                domain = -10:10,
                samples = 200,
                smooth,
                thick,
                red,
            ] {4*x - 2};
            \addplot[
                domain = -10:10,
                samples = 200,
                smooth,
                thick,
                green,
            ] {2*x - 2};
            \legend{
                4x - 2, 
                2x - 2,,   
            }
    \end{axis}
\end{tikzpicture}
\end{figure}
\section{Diferenciação parcial}
\subsection*{Introdução}
Vimos como derivar funções de uma variável e, talvez sem surpresa, também podemos diferenciar funções de duas variáveis usando a diferenciação parcial para produzir derivadas parciais.

\par De certa forma, isso será semelhante ao que vimos quando diferenciamos funções de uma variável para obter suas derivadas, mas como agora temos duas variáveis para lidar, as coisas ficam um pouco mais complicadas.
\par Considere uma função de duas variáveis independentes, $f(x,y)$.

Para um valor fixo de \textbf{y}, digamos $y = y\textsubscript{0}$, podemos olhar para a função $g(x) = f(x,y\textsubscript{0})$ que agora é uma função apenas de \textbf{x}. Claramente, a taxa de variação de $g(x)$ em relação a \textbf{x} é apenas a derivada dessa função em relação a \textbf{x}.
\par Mas, o que acontece quando queremos calcular a taxa de variação de $f(x,y)$ em relação a \textbf{x} para qualquer valor fixo de \textbf{y}? Para fazer isso, evitamos especificar um valor particular de \textbf{y} apenas assumindo que \textbf{y} é uma constante e diferenciando em relação a \textbf{x}.
\par Então, dada uma função $f(x,y)$ denotamos a operação de derivar $f$ em relação a \textbf{x} mantendo \textbf{y} constante por
\begin{equation}
    \frac{\partial f}{\partial x} \textrm{ ou, a alternativa }   f\textsubscript{x}(x,y)
\end{equation}
e chamamos isso de derivada parcial de $f(x,y)$ em relação a \textbf{x}.

\par De maneira semelhante, podemos definir a derivada parcial de $f(x,y)$ em relação a \textbf{y}, denotada por
\begin{equation}
    \frac{\partial f}{\partial y} \textrm{ ou, a alternativa }   f\textsubscript{y}(x,y)
\end{equation}
\par \textbf{Nota:} Usamos o "curly-d", ou seja, "$\partial$", para derivadas parciais em vez do "straight-d" normal, ou seja, "d", que se encontra na notação $\frac{dg}{dx}$ para a derivada de uma função $g(x)$ de uma variável. Veremos por que é importante manter essas duas noções de diferenciação separadas mais adiante.
\subsection*{Derivadas parciais e seções}
Claramente, a derivada parcial de $f(x,y)$ em relação a \textbf{x}, ou seja, o resultado da diferenciação de $f(x,y)$ em relação a \textbf{x} mantendo \textbf{y} constante, será outra função de \textbf{x} e \textbf{y} que temos chamado $fx(x,y)$. Mas, o que essa derivada parcial representa?
\par Com efeito, quando consideramos a função $f(x,y)$ para algum valor fixo de \textbf{y}, digamos \textbf{y\textsubscript{0}}, estamos olhando para a seção $y = y\textsubscript{0}$ da curva $z = f(x,y)$, ou seja, a seção dada por a equação $z = f(x,y\textsubscript{0})$ que se encontra em um plano que tem $y = y\textsubscript{0}$ e é paralelo ao plano \textit{\textbf{(x,z)}}.
\par Então, quando diferenciamos $f(x,y\textsubscript{0})$ em relação a \textbf{x}, estamos encontrando o gradiente desta seção, ou seja, ele nos diz como $z = f(x,y\textsubscript{0})$ está variando com \textbf{x}.
\par Consequentemente, essa derivada parcial está nos dizendo algo sobre o gradiente da superfície quando estamos no ponto $(x,y\textsubscript{0})$ e estamos "olhando" na direção \textbf{x}.
\subsection*{Encontrando derivadas parciais}
Calcular as derivadas parciais de $f(x,y)$ é apenas um pouco mais difícil do que encontrar a derivada de uma função de uma variável.

\par Lembrando que a derivada parcial de uma função $f(x,y)$ em relação a \textbf{x}, ou seja, $f\textsubscript{x}(x,y)$, é apenas a derivada de $f(x,y)$ em relação a \textbf{x} mantendo \textbf{y} constante, para calcular $f\textsubscript{x}(x,y)$ tratamos qualquer ocorrência de \textbf{y} em $f(x,y)$ como se fosse uma constante e diferenciamos em relação a \textbf{x}.
\par Dado que $f(x,y) = x\textsuperscript{2}y + 5xy\textsuperscript{3} + y\textsuperscript{2}$, encontraremos $f\textsubscript{x}(x,y)$ e $f\textsubscript{y}(x,y)$
\par Para encontrar $f\textsubscript{x}(x,y)$, tratamos \textbf{y} como se fosse uma constante e digamos que essa constante seja \textbf{c}. Então, temos uma função de uma variável dada por
\begin{equation}
    g(x) = f(x,c) = cx\textsuperscript{2} + 5c\textsuperscript{3}x + c\textsuperscript{2}
\end{equation}
e derivando isso em relação a \textbf{x} dá
\begin{equation}
    \frac{dg}{dx} = 2cx + 5c\textsuperscript{3}
\end{equation}
Mas, \textbf{c} é a constante que estamos usando para representar \textbf{y} e assim substituindo todos os \textit{'c's} por \textit{'y's} que temos
\begin{equation}
    \frac{\partial f}{\partial x} = 2xy + 5y\textsuperscript{3}
\end{equation}
que é a derivada parcial de $f(x,y)$ em relação a \textbf{x}.
\par Para encontrar $f\textsubscript{y}(x,y)$, tratamos \textbf{x} como se fosse uma constante e digamos que essa constante seja \textbf{c}. Então, temos uma função de uma variável dada por
\begin{equation}
    \begin{split}    
        g(y) = f(c,y)  & = c\textsuperscript{2}y + 5cy\textsuperscript{3} + y\textsuperscript{2} \\
        \frac{dg}{dy} & =  c\textsuperscript{2} + 15cy\textsuperscript{2} + 2y \\
        \frac{\partial f}{\partial x} & = x\textsuperscript{2} + 15xy\textsuperscript{2} + 2y
    \end{split}
\end{equation}
\subsection*{Derivadas parciais e as regras de diferenciação}
Até agora, calculamos as derivadas parciais de funções muito simples de \textbf{x} e \textbf{y}. Mas, às vezes, precisaremos usar as regras da cadeia, do produto e do quociente ao calcular as derivadas parciais.

\par Vejamos um exemplo para ver como isso é feito.
\par Dado que
\begin{equation}
    f(x,y) = xe\textsuperscript{x + y\textsuperscript{2}} 
\end{equation}
Primeiro notamos que podemos escrever esta função como
\begin{equation}
    f(x,y) = (xe\textsuperscript{x})e\textsuperscript{y\textsuperscript{2}}
\end{equation}
e assim, para encontrar $fx(x,y)$, tratamos $e\textsuperscript{y\textsuperscript{2}}$ como uma constante e derivamos a função $xe\textsuperscript{x}$ usando a regra do produto para obter $xe\textsuperscript{x} + 1e\textsuperscript{x}$. Isso nos dá
\begin{equation}
    \frac{\partial f}{\partial x} = e\textsuperscript{y\textsuperscript{2}}(xe\textsuperscript{x} + e\textsuperscript{x}) = (x + 1)e\textsuperscript{x + y\textsuperscript{2}}
\end{equation}
Para encontrar $f\textsubscript{y}(x,y)$, tratamos $xe\textsuperscript{x}$ como uma constante e derivamos a função $e\textsuperscript{y\textsuperscript{2}}$ usando a regra da cadeia para obter $2ye\textsuperscript{y\textsuperscript{2}}$. Isso nos dá
\begin{equation}
    \frac{\partial f}{\partial y} = xe\textsuperscript{x}(2ye\textsuperscript{y\textsuperscript{2}}) = 2xye\textsuperscript{x+y\textsuperscript{2}}
\end{equation}
\section{Regra da cadeia}
\subsection*{Introdução}
Às vezes, uma função de uma variável é definida com referência a uma função de duas variáveis.

\par Por exemplo, suponha que o nível de produção, \textbf{q}, de uma firma dependa dos montantes \textbf{k} de capital e \textbf{l} de trabalho usados por meio da função $q(k,l)$. Suponha também que \textbf{k} e \textbf{l} mudem ao longo do tempo de alguma forma conhecida, de modo que tenhamos fórmulas para $k(t)$ e $l(t)$ onde \textbf{t} é um parâmetro de medição de tempo.

\par Como podemos encontrar a taxa de variação da produção com o tempo?
\par Dado que temos a função de produção $q(k,l)=kl$ onde \textbf{k} e \textbf{l} são funções do tempo, \textbf{t}, dado por
\begin{equation}
    \begin{split}        
        k(t) & = 3 + 2t \\
        l(t) & = 10 + 3t
    \end{split}
\end{equation}
Neste caso, podemos calcular a produção em função do tempo encontrando explicitamente $Q(t) = q(k(t),l(t))$ que, neste caso, é
\begin{equation}
    \begin{split}        
        Q(t) & = k(t)l(t) \\
         & = (3 + 2t)(10 + 3t) \\
         & = 30 + 11t + 6t\textsuperscript{2}
    \end{split}
\end{equation}
E, em particular, agora podemos diferenciar isso para encontrar a taxa de variação da produção com o tempo, ou seja, temos
\begin{equation}
    \frac{dQ}{dt} = 11 - 12t
\end{equation}
Mais geralmente, suponha que nos seja dada uma função $f$ de duas variáveis \textbf{x} e \textbf{y}, ambas funções de \textbf{t}.
\par Podemos pensar nisso como definindo uma função composta $F(t) = f(x(t),y(t))$.
\par No caso de uma única variável, temos uma regra, ou seja, a regra da cadeia, que nos permite calcular a derivada de uma função composta. Surpreendentemente, talvez, exista uma regra semelhante para funções compostas de duas variáveis, como a que temos aqui, que também é conhecida como regra da cadeia. Diz que
\begin{equation}
    \frac{dF}{dx} = \frac{\partial f}{\partial x}\frac{dx}{dt} + \frac{\partial f}{\partial y}\frac{dy}{dt}
\end{equation}
Algumas vezes, neste contexto, chamamos $F'(t)$ de derivada total de $F(t)$ em relação a \textbf{t} (para distingui-la das derivadas parciais de $f$ em relação a \textbf{x} e \textbf{y}).
\subsubsection*{Regra de cadeia Exemplo}
Considere novamente a função de produção $q(k,l) = kl$ onde \textbf{k} e \textbf{l} são funções do tempo, \textbf{t}, dadas por
\begin{equation}
    \begin{split}        
        k(t) & = 3 + 2t \\
        l(t) & = 10 - 3t
    \end{split}
\end{equation}
Neste caso, se novamente fizermos $Q(t) = q(k(t),l(t))$, a regra da cadeia afirma que
\begin{equation}
    \begin{split}        
        \frac{dQ}{dt} & = \frac{\partial q}{\partial k}\frac{dk}{dt} + \frac{\partial q}{\partial l}\frac{dl}{dt} \\
        & = (l)(2) + (k)(-3) \\
        & = 2(10 - 3t) + 3(3 + 2t) \\
        & = 11 - 12t
    \end{split}
\end{equation}
\subsection{Funções explícitas}
Uma equação $g(x,y)=c$ onde \textbf{c} é uma constante pode, em alguns casos, ser rearranjada (ou resolvida) para dar \textbf{y} como uma função explícita de \textbf{x}. Uma vez feito isso, podemos então derivar nossa expressão para \textbf{y} em relação a \textbf{x} para encontrar sua derivada, $y'(x)$.
\par Mas, em geral, pode ser difícil ou impossível resolver a equação $g(x,y) = c$ para encontrar uma fórmula explícita para $y(x)$ como fizemos no exemplo anterior.

No entanto, podemos [muitas vezes] ainda encontrar a derivada $y'(x)$, mesmo que não tenhamos uma expressão explícita para \textbf{y} em termos de \textbf{x}.
\par Para ver como derivar uma função definida implicitamente, considere que se conhecêssemos a função, $y(x)$, que satisfizesse a equação $g(x,y) = c$, poderíamos encontrar uma nova função, $G(x)$, de \textbf{x} somente que seria dado por $G(x) = g(x,y(x))$. Então, usando a regra da cadeia, teríamos
\begin{equation}
    \frac{dG}{dx} = \frac{\partial g}{\partial x} \frac{dx}{dx} + \frac{\partial g}{\partial y} \frac{dy}{dx}
\end{equation}
Mas, $G(x) = c$ onde \textbf{c} é uma constante e assim também temos
\begin{equation}
    \frac{dG}{dx} = 0 \textrm{ assim como } \frac{dx}{dx} = 1
\end{equation}
o que significa que ficamos com
\begin{equation}
     0 = \frac{\partial g}{\partial x} + \frac{\partial g}{\partial y} \frac{dy}{dx} 
\end{equation}
Reorganizar isso nos dá
\begin{equation}
    \frac{dy}{dx} = - \frac{\partial g/\partial x}{\partial g/\partial y}
\end{equation}
contanto que $g\textsubscript{y}(x,y) \neq 0$
\par No exemplo anterior, \textbf{y} era uma função de \textbf{x} definida implicitamente pela equação $x\textsuperscript{2} - y = 7$. Encontramos $y'(x)$ usando o resultado acima.
\par Como temos a equação $x\textsuperscript{2} - y = 7$ podemos escrever isso como $g(x,y) = c$ com $g(x,y) = x\textsuperscript{2} - y$ e $c = 7$. Usando o resultado acima podemos ver que
\begin{equation}
    \frac{\partial g}{\partial x} = 2 \textrm{ e }\frac{\partial g}{\partial y} = - 1
\end{equation}
o que significa que
\begin{equation}
    \begin{split}
        \frac{dy}{dx} & = - \frac{\partial g/\partial x}{\partial g/\partial y} \\
        & = - \frac{2x}{-1} \\
        & = 2x
    \end{split}
\end{equation}
Suponha que \textbf{y} seja uma função de \textbf{x} definida implicitamente pela equação
\begin{equation}
    x\textsuperscript{2}y\textsuperscript{3} - 6x\textsuperscript{3}y\textsuperscript{2} + 2xy = 1
\end{equation}
Verifique se o ponto $(x,y) = (1/2,2)$ satisfaz esta equação e encontre o valor da derivada, $y'(x)$, neste ponto.
\par O ponto $(x,y) = (1/2,2)$ satisfaz a equação, pois, colocando $x = 1/2$ e $y = 2$ no lado esquerdo, obtemos
\begin{equation}
    \frac{1}{2}\textsuperscript{2}(2)\textsuperscript{3} - 6\textsuperscript{3}(2)\textsuperscript{3} - 6\frac{1}{2}\textsuperscript{3}(2)\textsuperscript{2} + 2\frac{1}{2}(2) = 1
\end{equation}
\par Vemos então que a equação que define \textbf{y} implicitamente como uma função de \textbf{x} é da forma $g(x,y) = 1$ onde $g(x,y) = x\textsuperscript{2}y\textsuperscript{3} - 6x\textsuperscript{3}y\textsuperscript{2}+2xy$. Então, de acordo com a fórmula dada acima, temos
\begin{equation}
    \begin{split}
        \frac{dy}{dx} & = - \frac{\partial g/\partial x}{\partial g/\partial y} \\
        \frac{\partial g}{\partial x}& = 2xy\textsuperscript{3} - 18x\textsuperscript{2}y\textsuperscript{2} + 2y \\
        \frac{\partial g}{\partial y}& = 3x\textsuperscript{2}y\textsuperscript{2} + 12x\textsuperscript{3}y + 2x \\
        \frac{dy}{dx} & = - \frac{2xy\textsuperscript{3} - 18x\textsuperscript{2}y\textsuperscript{2} + 2y}{3x\textsuperscript{2}y\textsuperscript{2} + 12x\textsuperscript{3}y + 2x}
    \end{split}
\end{equation}
contanto que $3x\textsuperscript{2}y\textsuperscript{2} + 12x\textsuperscript{3}y + 2x \neq 0$.
\par Assim, dado o ponto (1/2,2), podemos substituir esses valores em nossa expressão para $y'(x)$ para ver que o valor da derivada neste ponto é \textbf{6}.
\section{Extensões da Regra da Cadeia}
\subsection*{Introdução}
\par Suponha, por exemplo, que \textbf{g} é uma função de duas variáveis \textbf{x} e \textbf{y}, ambas elas próprias funções de duas variáveis \textbf{k} e \textbf{l}.
Podemos pensar nisso como a definição de uma função composta
\begin{equation}
    G(k,l) = g(x(k,l),y(k,l))
\end{equation}
e uma extensão da regra da cadeia nos garante que
\begin{equation}
    \begin{split}
        \frac{\partial G}{\partial k} & = \frac{\partial g}{\partial x} \frac{\partial x}{\partial k} + \frac{\partial g}{\partial y} \frac{\partial y}{\partial k}  \\
        \frac{\partial G}{\partial l} & = \frac{\partial g}{\partial x} \frac{\partial x}{\partial l} + \frac{\partial g}{\partial y} \frac{\partial y}{\partial l}
    \end{split}
    \end{equation}
Para ver por que a fórmula da regra da cadeia para $G\textsubscript{k}(k,l)$ funciona, considere que se mudarmos \textbf{k} por uma pequena quantidade, \textbf{$\Delta k$}, mantendo \textbf{l} constante, a variação correspondente em $G(k,l)$ é dada por
\begin{equation}
    \Delta G \simeq \frac{\partial G}{\partial k} \Delta k
\end{equation}
mas aqui, existem duas maneiras pelas quais $G(k,l) = g(x(k,l),y(k,l))$ pode mudar com \textbf{k}.
\begin{itemize}
    \item Em primeiro lugar, \textbf{G} pode mudar com \textbf{k} porque \textbf{g} muda com \textbf{x} e \textbf{x} muda com \textbf{k}, vamos denotar essa mudança em \textbf{G} por $\Delta \textsubscript{x}G$. Neste caso, temos
    \begin{equation}
        \Delta \textsubscript{x}G \simeq \frac{\partial g}{\partial x}\Delta x
    \end{equation}
    como estamos mantendo \textbf{y} constante para ver como \textbf{F} muda com \textbf{x} e isso significa que
    \begin{equation}
        \Delta \textsubscript{x}G \simeq \frac{\partial g}{\partial x} \frac{\partial x}{\partial k}\Delta k
    \end{equation}
    como a mudança em \textbf{x}, \textbf{$\Delta x$}, está relacionada a uma mudança em \textbf{k} com \textbf{l} mantido constante por $\Delta x \simeq x\textsubscript{k}(k,l)\Delta k$.
    \item Em segundo lugar, \textbf{G} pode mudar com \textbf{k} porque \textbf{g} muda com \textbf{y} e \textbf{y} muda com \textbf{k}, vamos denotar essa mudança em \textbf{G} por $\Delta \textsubscript{y}G$. Neste caso, temos
    \begin{equation}
        \Delta \textsubscript{y}G \simeq \frac{\partial g}{\partial y}\Delta y
    \end{equation}
    como estamos mantendo \textbf{x} constante para ver como \textbf{F} muda com \textbf{y} e isso significa que
    \begin{equation}
        \Delta \textsubscript{y}G \simeq \frac{\partial g}{\partial y} \frac{\partial y}{\partial k}\Delta k
    \end{equation}
    como a mudança em \textbf{y}, \textbf{$\Delta y$}, está relacionada a uma mudança em \textbf{k} com \textbf{l} mantido constante por $\Delta y \simeq y\textsubscript{k}(k,l)\Delta k$.
\end{itemize}
\par Assim, como a mudança total em \textbf{F} devido a essas duas mudanças é dada por
\begin{equation}
    \Delta G = \Delta\textsubscript{x}G + \Delta\textsubscript{y}G \simeq \frac{\partial g}{\partial x}\frac{\partial x}{\partial k}\Delta k + \frac{\partial g}{\partial y}\frac{\partial y}{\partial k}\Delta k
\end{equation}
agora podemos igualar nossas duas expressões para $\Delta G$ e dividir por $\Delta k$ para obter a regra da cadeia para $G\textsubscript{k}(k,l)$ que vimos acima.
\par E, de maneira semelhante, se supusermos que $g(x,y,z) = c$ define \textbf{z} implicitamente como uma função de \textbf{x} e \textbf{y}, podemos usar essa forma da regra da cadeia para derivar as fórmulas
\begin{equation}
    \begin{split}
        \frac{\partial z}{\partial x} = - \frac{\partial g/\partial x}{\partial g/\partial z} \\
        \frac{\partial z}{\partial y} = - \frac{\partial g/\partial y}{\partial g/\partial z}
    \end{split}
\end{equation}
contanto que $g\textsubscript{z}(x,y,z) \neq 0$.
\par Essas fórmulas nos permitirão calcular as derivadas parciais de \textbf{z} em relação a \textbf{x} e \textbf{y}.
\par De fato, para ver por que a fórmula para $z\textsubscript{x}(x,y)$ funciona, consideramos que se conhecêssemos a função, $z(x,y)$, que satisfizesse a equação $g(x,y,z) = c$, poderíamos encontrar uma nova função, $G(x,y)$, de \textbf{x} e \textbf{y} somente que é dada por
\begin{equation}
    G(x,y) = g(x,y,z(x,y))
\end{equation}
Então, usando a regra da cadeia, temos
\begin{equation}
    \begin{split}
        \frac{\partial G}{\partial x} & = \frac{\partial g}{\partial x} \frac{dx}{dx} + \frac{\partial g}{\partial z}\frac{\partial z}{\partial x}\\
        \frac{\partial G}{\partial y} & = \frac{\partial g}{\partial y} \frac{dx}{dx} + \frac{\partial g}{\partial z}\frac{\partial z}{\partial y}
    \end{split}
\end{equation}
Mas, $G(x,y) = c$ onde \textbf{c} é uma constante e então também temos
\begin{equation}
    \begin{split}
        \frac{\partial G}{\partial x} & = 0\\
        \frac{dx}{dx} & = 1 \\
        \frac{\partial G}{\partial x} & = 0\\
        \frac{\partial G}{\partial y} & = 0\\
        0 &= \frac{\partial g}{\partial x} + \frac{\partial g}{\partial z}\frac{\partial z}{\partial x}\\
        \frac{\partial z}{\partial x} & = - \frac{\partial g/\partial x}{\partial g/\partial z} \\
    \end{split}
\end{equation}
contanto que $g\textsubscript{z}(x,y,z) \neq 0$.
\par Ou seja, $z\textsubscript{x}(x,y)$ pode ser facilmente encontrado usando as derivadas parciais de \textbf{c}. \textbf{(Mas, não se esqueça do sinal de menos!)}
\subsubsection*{Exemplo}
Suponha que \textbf{q} seja uma função de \textbf{k} e \textbf{l} definida implicitamente pela equação
\begin{equation}
    q\textsuperscript{3}k + k\textsuperscript{3}l + qk\textsuperscript{2}l= 3
\end{equation}
Encontre as derivadas parciais $q\textsubscript{k}(k,l)$ e $q\textsubscript{l}(k,l)$. Quais são os valores dessas derivadas parciais no ponto onde $k = 1$ e $l = 1$?
\begin{equation}
    \begin{split}
        g(q,k,l) & = q\textsuperscript{3}k + k\textsuperscript{3}l + qk\textsuperscript{2}l  \textrm{ e } c = 3\\
        \frac{\partial q}{\partial k} &= - \frac{\partial g}{\partial k}/\frac{\partial g}{\partial q} \\
        \frac{\partial q}{\partial l} &= - \frac{\partial g}{\partial l}/\frac{\partial g}{\partial q} \\
        \frac{\partial q}{\partial k} &= - \frac{q\textsuperscript{3} + 3k\textsuperscript{2}l + 2qkl}{3q\textsuperscript{2}k + k\textsuperscript{2}l}\\
        \frac{\partial q}{\partial l} &= - \frac{k\textsuperscript{3} + qk\textsuperscript{2}}{3q\textsuperscript{2}k + k\textsuperscript{2}l}\\
    \end{split}
\end{equation}
desde que $3q\textsuperscript{2}k + k\textsuperscript{2}l \neq 0$.
\par Agora, para calcular essas derivadas parciais no ponto onde $(k,l) = (1,1)$, precisamos encontrar o valor correspondente de \textbf{q}. Isso pode ser feito observando que, quando temos $k = 1$ e $l = 1$, a equação se torna
\begin{equation}
    \begin{split}
        q\textsuperscript{3} + q - 2 &= 0\\
        (q - 1)(q\textsuperscript{2} + q + 2) &= 0
    \end{split}
\end{equation}
para todo $q \in R$, vemos que $q = 1$ é a única solução para esta equação. Assim, o ponto que nos interessa tem coordenadas $(k,l,q) = (1,1,1)$ e, neste ponto, temos como os valores das derivadas parciais:
\begin{equation}
    \begin{split}
        \frac{\partial q}{\partial k} &= - \frac{1 + 3 + 2}{3 + 1} = - \frac{6}{4} = - \frac{3}{2} \\
        \frac{\partial q}{\partial l} &= - \frac{1 + 1}{3 + 1} = - \frac{2}{4} = - \frac{1}{2}
    \end{split}
\end{equation}
\section{Funções Homogêneas e Teorema de Euler}
\subsection*{Introdução}
As funções homogêneas são importantes em economia, pois nos permitem capturar a ideia de "retornos à escala". Nesta seção veremos o que significa uma função ser homogênea e consideraremos um importante teorema sobre funções homogêneas. A primeira nos permitirá dar uma interpretação econômica das funções de produção homogêneas em termos de "retornos à escala" e a segunda nos permitirá considerar o significado econômico dos produtos marginais que podem ser derivados de tais funções de produção.
\par Dizemos que uma função, $f(x,y)$, é homogênea de grau $r$ se
\begin{equation}
    f(\lambda x,\lambda y) = \lambda\textsuperscript{r}f(x,y)
\end{equation}
\subsubsection*{Exemplo}
$f(x,y) = \sqrt[]{x} - \sqrt[]{y}$
\par Substituindo \textbf{x} e \textbf{y} em $f(x,y)$ por $\lambda x$ e $\lambda y$ temos
\begin{equation}
    f(\lambda x,\lambda y)=\sqrt[]{\lambda x} + \sqrt[]{\lambda y} = \sqrt[]{\lambda} \sqrt[]{x} + \sqrt[]{\lambda} \sqrt[]{y} = \lambda(\sqrt[]{x} + \sqrt[]{y}) = \sqrt[]{\lambda}f(x,y)
\end{equation}
Como $\sqrt[]{\lambda} = \lambda\textsuperscript{1/2}$, comparando isso com a definição de homogeneidade, ou seja.
\begin{equation}
    f(\lambda x,\lambda y) = \lambda\textsuperscript{r}f(x,y)
\end{equation}
vemos que $r = 1/2$ e assim esta função é homogênea de grau meio.
\subsubsection*{Exemplo 2}
Substituindo x e y em $f(x,y)$ por $\lambda x$ e $\lambda y$ temos
\begin{equation}
    f(\lambda x,\lambda y) = (\lambda x) + (\lambda y)\textsuperscript{2} = \lambda x + \lambda\textsuperscript{2}y\textsuperscript{2}.
\end{equation}
Comparando isso com a definição de homogeneidade, ou seja, $f(\lambda x,\lambda y) = \lambda\textsuperscript{r}f(x,y)$ vemos que não há como escrever $\lambda x + \lambda\textsuperscript{2}y\textsuperscript{2}$ na forma $\lambda\textsuperscript{r}(x+y2)$ para qualquer r e, portanto, essa função não é homogênea.

\par Em particular, isso significa que nem todas as funções são homogêneas.
\subsection*{Retornos à Escala}
Economicamente, podemos pensar em funções homogêneas como nos dizendo sobre como as saídas mudam se "aumentarmos" nossas entradas. Para ver por que, considere o que acontece se escalarmos nossas entradas por um fator de $\lambda$, ou seja, se aumentarmos nosso pacote de entradas, $(x,y)$, por um fator de $\lambda > 1$, obtemos o novo pacote de entradas $(\lambda x ,\lambda y)$.

Agora, se nossas saídas são determinadas por uma função homogênea, $f(x,y)$, de grau $r$, podemos ver que a saída de nosso novo pacote, $(\lambda x,\lambda y)$, é dada por
\begin{equation}
    f(\lambda x,\lambda y) = \lambda\textsuperscript{r}f(x,y)
\end{equation}
ou seja, obteremos $\lambda\textsuperscript{r}$ vezes tanto quanto obtivemos do nosso pacote antigo, $(x,y)$. Ou seja, escalonar as entradas por $\lambda$ leva a um escalonamento da saída por $\lambda\textsuperscript{r}$ se nossa saída for determinada por uma função homogênea de grau $r$.

\par Em particular, dada uma função que é homogênea de grau um, podemos ver que escalar nossas entradas por $\lambda > 1$  ou seja, indo do pacote de entradas $(x,y)$ para o pacote de entradas $(\lambda x,\lambda y)$ irá escalar nossa saída por $\lambda$ ou seja, indo de uma saída de $f(x,y)$ para uma saída de $\lambda f(x,y)$. Ou seja, obtemos \textit{retornos constantes de escala}, um aumento proporcional nos insumos leva ao mesmo aumento proporcional na produção.

\par Claramente, dadas funções de grau $r > 0$, esta ideia pode ser estendida para cobrir funções com graus $r \neq 1$ como segue.
\begin{itemize}
    \item Se $r > 1$, obtemos retornos crescentes de escala, pois $\lambda > 1$ implica que $\lambda\textsuperscript{r} > \lambda$. Ou seja, um aumento proporcional nos valores de entrada leva a um aumento proporcional maior nos valores de saída.
    \item Se $r = 1$, obtemos retornos \textit{constantes} à escala como vimos acima.
    \item Se $r < 1$, obtemos retornos decrescentes de escala, pois $\lambda > 1$ implica que $\lambda\textsuperscript{r} < \lambda$. Ou seja, um aumento proporcional nos valores de entrada leva a um aumento proporcional menor nos valores de saída.
\end{itemize}
\subsubsection*{Exemplo}
Uma empresa investe uma quantidade de capital, \textbf{k}, e trabalho, \textbf{l}, em seu processo de produção e isso produz um nível de produção de $q(k,l)$.

\par Qual será o efeito sobre o nível de produção de quadruplicar a quantidade de capital e trabalho investido se a função de produção for homogênea de grau \textit{(a) 1/2, (b) 1 e (c) 3/2}?
\par Quadruplicar a quantidade de capital e trabalho investido significa aumentar a cesta de investimentos de $(k,l)$ para $(4k,4l)$. Assim, se a função de produção for homogênea de grau r, o nível de produção irá de $q(k,l)$ para
\begin{equation}
    q(4k,4l) = \lambda\textsuperscript{r}q(k,l)
\end{equation}
Em particular, isso significa que se a função de produção for homogênea de grau
\begin{enumerate}[label = \alph*)]
    \item 1/2, a mudança será por um fator de $4\textsuperscript{1/2} = 2$ (ou seja, quadruplicar os insumos duplica a produção)
    \item 1, a mudança será por um fator de $4\textsuperscript{1} = 4$ (ou seja, a quadruplicação dos insumos quadruplica a produção)
    \item 3/2, a mudança será por um fator de $4\textsuperscript{3/2} = 8$ (ou seja, a quadruplicação de insumos óctupla a produção)
\end{enumerate}
gerando retornos decrescentes, constantes e crescentes de escala, respectivamente.
\subsection*{Teorema de Euler}
Agora nos voltamos para o chamado teorema de Euler, que diz o seguinte.

\par Se $f(x,y)$ é uma função homogênea de grau r, então
\begin{equation}
    x\frac{\partial f}{\partial x} + y\frac{\partial f}{\partial y} = rf(x,y)
\end{equation}
\subsubsection*{Exemplo}
A função $f(x,y) = x\textsuperscript{1/2}y\textsuperscript{1/2}$ é homogênea de grau um.
\par Vamos verificar se o teorema de Euler vale para esta função.
\begin{equation}
    \begin{split}
        \frac{\partial f}{\partial x} &= \frac{1}{2}x\textsuperscript{-1/2}y\textsuperscript{1/2} \\
        \frac{\partial f}{\partial y} &= \frac{1}{2}x\textsuperscript{1/2}y\textsuperscript{-1/2} \\
        x\frac{\partial f}{\partial x} + y\frac{\partial f}{\partial y} &= x(\frac{1}{2}x\textsuperscript{-1/2}y\textsuperscript{1/2}) + y(\frac{1}{2}x\textsuperscript{1/2}y\textsuperscript{-1/2})\\
        &= \frac{1}{2}x\textsuperscript{1/2}y\textsuperscript{1/2} + \frac{1}{2}x\textsuperscript{1/2}y\textsuperscript{1/2} \\
        &= x\textsuperscript{1/2}y\textsuperscript{1/2} \\
        &= f(x,y) 
    \end{split}
\end{equation}
e como o grau de homogeneidade desta função é um, temos $f(x,y)$ no lado direito do teorema de Euler. Assim, como essas duas expressões são iguais, vale o teorema de Euler.
\subsection*{O significado econômico do teorema de Euler}
Considere uma empresa que investe uma quantidade de capital, k, e trabalho, l, em seu processo de produção e isso produz um nível de produção de $q(k,l)$.

\par Além disso, suponha que essa função de produção seja homogênea de grau um, ou seja, que tenhamos retornos constantes de escala. O teorema de Euler afirma então que
\begin{equation}
        k\frac{\partial q}{\partial k} + l\frac{\partial q}{\partial l} = q(k,l).
\end{equation}
Agora, $q\textsubscript{l}$ nos dá o produto marginal do trabalho, ou seja, mede a mudança na produção se mudarmos a quantidade de trabalho. Em particular, se usarmos mais uma unidade de trabalho, digamos, empregando mais um trabalhador, $q\textsubscript{l}$ nos diz a mudança resultante na produção.

\par Assim, faz sentido dizer que esse trabalhador extra é responsável por essa mudança na produção e, portanto, se assumirmos que recompensamos os trabalhadores dando-lhes bens iguais à quantidade que produzem, faz sentido recompensar esse trabalhador com uma quantidade de bens dados por $q\textsubscript{l}$.

\par Assim, se todos os trabalhadores produzem a mesma quantidade, ou seja, $q\textsubscript{l}$, e existem $l$ (ou seja, a quantidade de trabalho utilizada) trabalhadores, faz sentido que todos sejam recompensados com uma quantidade de bens igual a $q\textsubscript{l}$. Como tal, $l\textsubscript{ql}$ representa a quantidade total de bens que devem ser dados como recompensa aos trabalhadores (ou seja, o trabalho).

\par Um argumento semelhante se aplica à quantidade $k\textsubscript{qk}$, ou seja, esta deve ser a quantidade total de bens que deve ser dada como recompensa aos provedores do capital.

\par Consequentemente, o teorema de Euler nos diz que essas recompensas devem somar a quantidade total de bens produzidos, ou seja, todos os bens produzidos serão distribuídos entre os fornecedores de capital e os fornecedores de trabalho.

\par Em resumo, isso diz o seguinte.

\par Em uma empresa com retornos constantes de escala, se recompensarmos cada fator de produção (por exemplo, capital e trabalho) em um nível igual ao seu produto marginal, então a recompensa total para os fatores de produção será a quantidade produzida.
\section{Derivativos parciais de segunda ordem}
\subsection*{Introdução}
\par Se tivermos uma função $f(x,y)$, podemos usar a diferenciação parcial para encontrar as novas funções $f\textsubscript{x}(x,y)$ e $f\textsubscript{y}(x,y)$. Essas novas funções são chamadas de derivadas parciais de primeira ordem de $f$. No entanto, também é possível diferenciar parcialmente essas novas funções em relação a \textbf{x} e \textbf{y} para obter as derivadas parciais de segunda ordem de $f$. Obviamente, para uma função de duas variáveis, existem \textit{\textbf{quatro derivadas parciais}} de segunda ordem, ou seja, aquelas que são "não misturadas"
\begin{equation}
    \begin{split}
        \frac{\partial\textsuperscript{2}f}{\partial x\textsuperscript{2}} & = \frac{\partial}{\partial x}(\frac{\partial f}{\partial x}) \\
        \frac{\partial\textsuperscript{2}f}{\partial y\textsuperscript{2}} & = \frac{\partial}{\partial y}(\frac{\partial f}{\partial y}) 
    \end{split}
\end{equation}
e as que são "mistas", 
\begin{equation}
    \begin{split}
        \frac{\partial\textsuperscript{2}f}{\partial y \partial x} & = \frac{\partial}{\partial y}(\frac{\partial f}{\partial x}) \\
        \frac{\partial\textsuperscript{2}f}{\partial x \partial y} & = \frac{\partial}{\partial x}(\frac{\partial f}{\partial y}) \\
        f\textsubscript{xx} &= (f\textsubscript{x})\textsubscript{x} \\
        f\textsubscript{yy} &= (f\textsubscript{y})\textsubscript{y} \\
        f\textsubscript{xy} &= (f\textsubscript{x})\textsubscript{y} \\
        f\textsubscript{yx} &= (f\textsubscript{y})\textsubscript{x} 
    \end{split}
\end{equation}
Neste curso, descobriremos que a ordem de diferenciação parcial nas derivadas parciais mistas de segunda ordem não é importante, pois sempre teremos $f\textsubscript{xy} = f\textsubscript{yx}$. Em particular, esse fato pode servir como uma "verificação" útil quando estamos trabalhando com derivadas parciais de segunda ordem.
\subsubsection*{Exemplo}
Derivadas de \textbf{primera ordem} de $f(x,y) = x\textsuperscript{2}y + 5xy\textsuperscript{3} + y\textsuperscript{2}$ são:
\begin{equation}
    \begin{split}
        f\textsubscript{x}(x,y) & = 2xy + 5y\textsuperscript{3} \\
        f\textsubscript{y}(x,y) & = x\textsuperscript{2} + 15xy\textsuperscript{2} + 2y \\
    \end{split}
\end{equation}
Dderivadas de \textbf{segunda ordem}:
\begin{equation}
    \begin{split}
        f\textsubscript{xx}(x,y) & = 2y \\
        f\textsubscript{yy}(x,y) & = 30xy + 2 \\
        f\textsubscript{xy}(x,y) & = 2x + 15y\textsuperscript{2} \\
        f\textsubscript{yx}(x,y) & = 2x + 15y\textsuperscript{2}
    \end{split}
\end{equation}
\subsubsection*{Exemplo 2}
Derivadas de \textbf{primera ordem} de $f(x,y) = 3x\textsuperscript{3} + 7xy\textsuperscript{-1} + 2y\textsuperscript{9}$ são:
\begin{equation}
    \begin{split}
        f\textsubscript{x}(x,y) & = 9x\textsuperscript{2} + 7y\textsuperscript{-1} \\
        f\textsubscript{y}(x,y) & = -7xy\textsuperscript{-2} + 18y\textsuperscript{8} \\
    \end{split}
\end{equation}
Dderivadas de \textbf{segunda ordem}:
\begin{equation}
    \begin{split}
        f\textsubscript{xx}(x,y) & = 18x \\
        f\textsubscript{yy}(x,y) & = 14xy\textsuperscript{-3} + 144y\textsuperscript{7} \\
        f\textsubscript{xy}(x,y) & = -7y\textsuperscript{-2} \\
        f\textsubscript{yx}(x,y) & = -7y\textsuperscript{-2}
    \end{split}
\end{equation}
\end{document}
